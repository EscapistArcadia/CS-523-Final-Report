\section{Evaluation}

In this section, we evaluate the performance of our implementation of the flattened page table in Linux. We first present the experimental setup, including the hardware and software configurations. Then, we compare the performance of our implementation with the original 4-level radix paging mechanism, and the ECPT paging mechanism in Linux using various benchmarks. Finally, we analyze the results and discuss the implications of our findings.

% TODO: add reference for ECPT, csynb

\subsection{Methodology and Metrics}

To evaluate the performance of our Flattened Page Table (FPT) implementation, we designed a comprehensive evaluation pipeline that executes benchmarks inside QEMU virtual machines, collects low-level memory traces, and performs offline analysis using instrumentation and custom tooling. Our methodology ensures consistency and fairness when comparing FPT against the baseline 4-level Radix Tree paging and the ECPT mechanism.

We run various benchmarks inside QEMU virtual machines with different configurations, including the original 4-level radix paging, the ECPT mechanism, and two configurations, L3L2 mode and L4L3+L2L1 mode, of our FPT implementation. The benchmarks include a mix of workloads that stress different aspects of the memory management system, such as memory allocation, page table lookups, and memory-intensive applications. For each benchmark, we split the execution into loading and running phases. The loading phase is used to load the benchmark into memory, while the running phase is used to execute the benchmark. We measure the execution time of each phase separately to understand the impact of our FPT implementation on both loading and running times.

After executing the benchmarks, we collect low-level memory traces using DynamoRIO, a dynamic binary instrumentation framework. We collect information about memory accesses, page walk latencies, TLB misses, instruction counts, and some other performance metrics. We then generate some visualization graphs to help partition the memory access of each execution of benchmarks and compare the performance of the different configurations.

\subsection{Results}


