\section{Related Work}

Address translation has long been a fundamental challenge in virtual memory systems, especially for memory-intensive workloads with large working sets. Traditional x86\_64 systems use multi-level radix page tables, which, while space-efficient, suffer from translation overhead due to deep page walks. Numerous alternative designs have been proposed to reduce this overhead, improve scalability, or exploit memory-level parallelism.

To address the inefficiency of multi-level walks, several works have also proposed flattened page table structures. Park et al advocate flattening to improve TLB reach and page table cacheability, reducing page walk latency \cite{every_walks_a_hit}. Our Flattened Page Table (FPT) design extends this idea by selectively merging multiple page table levels (e.g., L3 and L2) to reduce walk depth while preserving the logical hierarchy.

Other works have explored alternative page table structures, like the clustered page table by Talluri and Hill \cite{clustered}, which groups page table entries to reduce the number of memory accesses during translation. In this structure, each entry in the hash table maps to a cluster of contiguous virtual pages, allowing multiple page table entries to be stored within a single hash table entry. This approach is particularly helpful for systems with sparse address spaces, as it reduces the overhead associated with managing numerous individual page table entries.

More recently, Skarlatos et al. proposed Elastic Cuckoo Page Tables (ECPT) \cite{ecpt}, which transform the inherently sequential radix walk into a parallel cuckoo hash lookup. By employing elastic cuckoo hashing, ECPTs support process-private page tables, multiple page sizes, and gradual resizing. This structure enables memory-level parallelism during translation.
