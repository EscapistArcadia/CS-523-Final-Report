\section{Related Work}

Address translation has long been a fundamental challenge in virtual memory systems, especially for memory-intensive workloads with large working sets. Traditional x86\_64 systems use multi-level radix page tables, which, while space-efficient, suffer from translation overhead due to deep page walks. Numerous alternative designs have been proposed to reduce this overhead, improve scalability, or exploit memory-level parallelism.

To mitigate the inefficiencies of deep page walks, several works have also proposed flattened page table structures. Park et al advocate flattening to improve TLB reach and page table cacheability, reducing page walk latency \cite{every_walks_a_hit}. Our Flattened Page Table (FPT) design extends this idea by selectively merging multiple page table levels (e.g., L3 and L2) to reduce walk depth while preserving the logical hierarchy.

Beyond flattening, researchers have explored alternative structures for page tables. Other works have explored alternative page table structures, like the clustered page table by Talluri and Hill \cite{clustered}, which groups multiple contiguous virtual pages into a single page table entry. By doing so, it reduces memory accesses for translations, making it particularly effective in systems with sparse or fragmented address spaces.

More recently, Skarlatos et al. proposed Elastic Cuckoo Page Tables (ECPT) \cite{ecpt}, a novel design that replaces sequential pointer-chasing with fully parallel lookups using cuckoo hashing. ECPT supports dynamic resizing, process-private page tables, and multiple page sizes. It exploits memory-level parallelism during address translation and achieves substantial performance gains by avoiding the serialized nature of radix tree walks.

These alternative designs collectively reflect a growing interest in rethinking traditional paging mechanisms to meet the demands of modern high-performance and data-intensive applications.
